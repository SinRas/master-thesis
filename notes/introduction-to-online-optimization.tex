% ---------------------------------------
% Reference Parameters
% Author: Sebastien Bubeck
% Title: Introduction to Online Optimization
% Publisher: Princeton University - Department of Operations Research and Financial Engineering
% Date: December 14, 2011

% ---------------------------------------
% Online Optimization
\subsubsection{
بهینه‌سازی برخط
}


% ---------------------------------------
% Introduction
\subsubsubsection{
مقدمه
}

در دنیایی که جمع‌آوری خودکار داده فراگیر شده است، آماردان‌ها باید چهارچوب‌های فکری خودشان را با مسئله‌های جدید مطابقت دهند. هر زمان که صحبت از شبکه‌ی اینترنت، داده‌های مشتری
\LTRfootnote{
Consumer Data
}
، و یا بازار مالی
\LTRfootnote{
Financial Market
}
 با میان می‌آید، یک ویژگی مشترک به چشم می‌آید: حجم بسیار زیاد داده که نیازمند فهم و تحلیل سریع است. شرایط مسئله‌های کنونی با آمار کلاسیک بسیار تفاوت دارد، تعداد مشاهده‌ها بسیار زیاد و تعداد متغیرهای مرتبط بسیار کم است. یکی از مدل‌های به تفصیل بررسی شده برای یادگیری، چهارچوب نظریه‌ی یادگیری آماری
 \LTRfootnote{
 Statistical Learning Theory
 }
  است. که به اختصار در ادامه بررسی خواهد شد:

% Definition
\begin{quote}
{\bf
پروتکل
\LTRfootnote{
Protocol
}
 یادگیری آماری:
} 
پروتکل ابتدایی یادگیری آماری به شکل زیر است:
\begin{itemize}
\item
مشاهد کنید نمونه‌های 
$Z_1, \cdots, Z_n \in \randomvariable{Z}$. 
فرض می‌کنیم که توزیع مشاهده‌ها مستقل و هم‌توزیع
\LTRfootnote{
I.I.D.
}
 از توزیع احتمال 
$\hollow{P}$
آمده اند.

\item
تصمیم بگیرید (یا عملی را انتخاب کنید) 
$a( Z_1, \cdots, Z_n ) \in \set{A}$ 
که 
$\set{A}$ 
یک مجموعه‌ی داده شده از تصمیم‌های ممکن است.

\item
به اندازه‌ی متوسط ضرر
\LTRfootnote{
Loss
} 
$\hollow{E}_{Z \sim \hollow{P}} \mathfunction{l}( a(Z_1,\cdots,Z_n), Z )$ 
هزینه بپردازید، که در آن 
$\mathfunction{l} : \set{A} \times \randomvariable{Z} \rightarrow \hollow{R}_+$ 
تابع ضرر داده شده است.
\end{itemize}

{\bf
هدف:
} 
کمینه کردن (و کنترل) ریسک مازاد
\LTRfootnote{
Excess Risk
}
:
\[
r_n = \hollow{E}_{Z \sim \hollow{P}} \mathfunction{l}( a(Z_1, \cdots, Z_n), Z ) - \inf_{a \in \set{A}} \hollow{E}_{Z \sim \hollow{P}} \mathfunction{l} ( a, Z )
\]
که معیار اندازه‌گیری میزان متوسط ضرر متحمل شده در مقایسه با انتخاب بهینه است.

\end{quote}



% Remark
\begin{quote}
{\bf
تعریف:
} 
کنترل کردن ریسک مازاد، به معنی پیدا کردن کران بالا برای 
$r_n$ 
است که یا در امید ریاضی
\LTRfootnote{
Expectation
} 
صدق کند ( نسبت به دنباله‌ی 
$Z_1,\cdots,Z_n$ 
) و یا با احتمال حداقل 
$1 - \delta$. 
معمولا کران بالا به شکلی عبارتی است بر اساس معیارهایی از پیچیدگی
\LTRfootnote{
Complexity Measure
} 
$\set{A}$ 
و 
$\mathfunction{l}$. 
همچنین اگر کران بالا به توزیع 
$\hollow{P}$ 
بستگی داشته باشد، می‌گوییم کران بالا وابسته به توزیع
\LTRfootnote{
Distribution-Dependent
} 
است و در غیر این صورت گوییم که آزاد از توزیع
\LTRfootnote{
Distribution-Free
} 
است.

\end{quote}


فرمول‌بندی بالا کلّی بوده و باعث می‌شود که مسئله‌های بسیاری را در بر گیرد. در ادامه به چند مثال اشاره می‌کنیم:

% Example
{\bf
مثال: تخمین رگرسیون
\LTRfootnote{
Regression Estimation
}
} 
\begin{itemize}
\item
داده‌ی مشاهده شده جفت‌هایی از نقطه‌ها به شکل 
$ Z_i = ( X_i, Y_i ) \in \set{X} \times \set{Y} $ 
هستند.


\item
مجموعه‌ی ممکن انتخاب‌ها، مجموعه‌ی تابع‌های از 
$\set{X}$ 
به 
$\set{Y}$ 
است، 
$\set{A} \subset \{ \mathfunction{f} : \set{X} \rightarrow \set{Y} \} $.


\item
تابع هزینه 
$\mathfunction{l}(a, (x,y) )$ 
میزان دقت پیش‌بینی تابع 
$\mathfunction{a}: \set{X} \rightarrow \set{Y}$ 
را اندازه‌گیری می‌کند. به عنوان نمونه اگر 
$\set{Y}$ 
یک فضای نرم‌دار
\LTRfootnote{
Normed Space
}
 باشد، یک انتخاب معمول 
$\mathfunction{l}(a, (x,y)) = || a(x) - y ||$ 
خواهد بود.

\end{itemize}

% Example
{\bf
تکمیل ماتریس (یا فیلتر کردن همکارانه!؟؟)
}




























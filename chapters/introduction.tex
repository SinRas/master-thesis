% !TEX encoding = UTF-8 Unicode


% ---------------------------------------
% Introduction
\section{
انگیزه و پیش‌نیازها
}

در 
\textbf{
قسمت اول
} 
به بیان انگیزه از انجام این پژوهش و ارائه‌ی دغدغه‌های نویسنده می‌پردازیم. تلاش شده است که خواننده را نیز به زمینه‌ی مورد بررسی علاقه‌مند کرد و میزان تاثیرگذاری 
\textit{
یادگیری برخط
} 
و 
\textit{
رانش مفهوم
} 
تبیین شود.

\textbf{
قسمت دوم
} 
به بیان دو مسئله‌ی عملی که مورد بررسی قرار گرفته‌اند پرداخته می‌شود. ابتدا مسئله به شکل کلامی و کلّی اما واضح آورده شده و پرسشی که به دنبال پاسخ آن هستیم، بیان می‌شود. سپس یک روش مدل کردن مسئله (که ادعا نمی‌شود بهترین روش مدل کردن مسئله است) که در ادامه قرار است مورد تحلیل بیش‌تر قرار بگیرد، ذکر می‌شود. چالش‌های در برخورد با مسئله پس از آن آورده شده است که در انتهای پایان‌نامه در رابطه با میزان پاسخ‌گویی به این چالش‌ها بیش‌تر بحث خواهد شد.

\textbf{
قسمت سوم
} 
پیش‌نیازهایی است که برای مطالعه، راحتی و فهم ادبیات و نمادگذاری این نوشته نیاز است. تلاش شده است که بیان، نمادگذاری و تعریف‌ها تا جای ممکن، در این نوشته، خودکفا باشند و نیاز به پیش‌نیاز خاصی که آورده نشده، نباشد. به این منظور قسمت سوم شامل 
\textit{
نمادگذاری زمان اجرای الگوریتم‌ها
}،
\textit{
بهینه‌سازی (محدب)
}، 
\textit{
بهینه‌سازی برخط و یادگیری برخط
} 
و 
\textit{
رانش مفهوم
} 
تعریف‌ها است.


% ---------------------------------------
% On the first page you should present:
% - the area of research ( e.g. implementation of information systems )
% - the most relevant previous findings in this area
% - your research problem and why this is worthwhile studying
% - the objective of the thesis: how far you hope to advance knowledge in the field


% ---------------------------------------
% Target Group
% to whom you are writing, what do you assume that the reader know? A normal target group would be new Master students.


% ---------------------------------------
% Personal Motivation
% why did you choose this topic?


% ---------------------------------------
% Research method in brief
% how will you find out?


% ---------------------------------------
% Structure of the report
% a paragraph about each chapter. What is the main contribution of the chapter? How do they relate?



% ---------------------------------------
% --> Motivation
\subsection{
انگیزه
}

هر موجود زنده‌ی دارای قدرت تفکر و اختیار، با برقراری ارتباط با محیط اطراف خود، اطلاعاتی را دریافت می‌کند. سپس از سیستم فکری و استنتاجی خود برای نتیجه‌گیری از اطلاعات دریافتی استفاده می‌کند تا برخوردهای بعد با محیط اطراف را تغییر دهد. این روندی است که باعث می‌شود دوچرخه‌سواری با بیاموزیم، از تجربه‌های تلخ (یا شیرین)، به اصطلاح، 
\textit{
درس بگیریم
} 
و عملکرد آکادمیک خود را بهبود بخشیم.

بگذارید باقی این قسمت را با یک مثال ادامه دهیم. فرض کنید که شما یک محقق هستید. اگر هزاران سال پیش به دنیا آمده و مشغول به فعالیت علمی بودید، احتمالا جزو معدود انسان‌های متفکر (در زمینه‌ی خود) بودید. این باعث می‌شد که مسئله‌ی محدود بودن زمان برای مطالعه برای شما مطرح نشود زیرا میزان فعالیت‌های آکادمیک آنقدر کم بوده که نیازی به عجله در دنبال کردن و مطالعه حس نمی‌کردید. اما امروزه در زمینه‌های آکادمیک هر روزه تعداد زیادی نوشته‌ی علمی منتشر می‌شود. اکنون اگر بخواهید به روز بمانید، این کار را به تنهایی نمی‌توانید در بیش از چند زمینه انجام دهید. به طور مثال شما می‌توانید از چند نفر (که هنوز به آن‌ها اطمینان کامل ندارید) در زمینه‌های مختلف بپرسید که نوشته‌های علمی در زمینه‌ی‌شان را به شما معرفی کنند. شما آن‌ها را بخوانید و اگر از نوشته‌هایی که یک فرد به شما فرستاده است در طول بازه‌ی زمانی رضایت نداشتید، دیگر از او درخواست نکنید. شاید کمی مثال دور کننده به نظر برسد اما این روش ساده، همانند مشورت گرفتن از افراد مختلف در زندگیتان، یکی از روش‌های مورد بررسی در یادگیری برخط است.

امروزه با رسیدن حجم زیادی از داده به شکل جریان
\LTRfootnote{
Stream
} 
نیازمند این هستیم که استنتاج و تصمیم‌گیری (به طور مثال یادگیری و پیش‌بینی) را برخط انجام دهیم. در گذشته (به دلیل مشابه با مثال بالا) این مسئله چندان مورد توجه قرار نگرفته بوده است. معمولا داده‌ها یکجا وجود داشته‌اند، تحلیل و یادگیری روی آن‌ها صورت می‌گرفت و سپس قدم بعد پیش می‌رفت. در این نوع تحلیل تمام داده در اختیار است و استدلال با توجه به این فرض انجام می‌شود.

در آخر به چند کاربرد عملی اشاره می‌کنیم. توجه داشته باشید که (تقریبا) هر مسئله‌ای که در آن داده به شکل گام‌به‌گام به شما می‌رسد را می‌توان جزو این مثال‌ها قرار داد.
***مثال‌ها رو رفرنس بده به مقاله‌هایی که استفاده کردن***
\begin{itemize}
\item \textbf{
پیش‌نهاد به یک مشتری
}
\item \textbf{
یادگیری شبکه‌های عصبی
}
\item \textbf{
تصمیم‌گیری با استفاده از نظر متخصص‌ها
}
\item \textbf{
تشخیص ای‌میل‌های واقعی از اسپم
}

\item \textbf{
پیش‌بینی قسمت سهام
}
\item \textbf{
***
}
\end{itemize}



% ---------------------------------------
% --> Problems
\subsection{
مسئله‌های مورد بررسی
}
برای بررسی روش‌ها و گرفتن انگیزه‌های عملی برای تحلیل‌های نظری، به آزمایش با دو مسئله پرداخته شده است. این دو مسئله شباهت‌ها و تفاوت‌هایی دارند که هریک در تحلیل روش و برداشتن گام‌های بعدی به ما کمک کرده است. در رابطه با ویژگی‌های مختلف در ادامه‌ی این نوشته بیش‌تر و دقیق‌تر صحبت خواهیم کرد.

% ---------------------------------------
% --> Traffic
\subsubsection{
مسئله‌ی ترافیک
}

مسئله، تخمین زمان سفر
\LTRfootnote{
Estimated Time of Arrival (ETA)
}
میان مبداء و مقصد از پیش مشخص شده در یک شهر است. قرار بر این است که این تخمین در زمانی کوتاه و با دقتی مناسب صورت بگیرد. به منظور قرار دهی مسئله در چهارچوب یادگیری برخط
\LTRfootnote{
Online Learning
}
ابتدا به صورت‌بندی آن به زبان ریاضی می‌پردازیم.

اطلاعات ما از وضعیت کنونی شهر در گرافی وزن‌دار
\LTRfootnote{
Weighted Graph
}
$G_t = ( \set{V}, \set{E}, \mthfnc{W}_t )$
خلاصه می‌شود که در آن اندیس
$t$
نمایانگر زمان کنونی،
$\set{V}$
و
$\set{E}$
به ترتیب مجموعه‌ی راس‌ها (تقاطع‌ها) و یال‌ها (اتصال‌های میان تقاطع‌ها) و
$\mthfnc{W}_t: \set{E} \rightarrow \hollow{R}ـ+$
وزن‌های روی هر یال است. جفت متمایز مبداء-مقصد را با دوتایی
$(src, dst) \in \set{V} \times \set{V}$
نشان دهید. تمامی مسیرهای ساده
\LTRfootnote{
Simple Paths
}
میان مبداء و مقصد را با
$\set{P}(src,dst) = \{ p_i \}^{k}_{i=1}$
نشان داده و اندیس گذاری می‌کنیم. تابع وزن مسیر به شکل زیر تعریف می‌شود:
\[
\mthfnc{T}_t(p) = \sum_{e \in p} \mthfnc{W}_t(e)
\]
هدف یافتن وزن‌های مناسب برای هر مسیر میان مبداء و مقصد است که تخمین مناسب‌تری از زمان سفر را به ما بدهد
\[
\mbox{find } x_{t-1} \in \simplex{k} \mbox{ s.t. } \sum^{k}_{i=1} x_{t-1}(i) * \mthfnc{T}_t(p_i) \approx \min_{i = 1, \cdots, k} \mthfnc{T}_t(p_i)
\]
مسئله به بیان یادگیری به کمک توصیه‌ی متخصص‌ها
\LTRfootnote{
Experts' Advice
}:
به ازای هر مسیر میان مبداء و مقصد، یک متخصص
\LTRfootnote{
Expert
}
داریم که همواره پیش‌بینی ثابتی دارد. هدف یافتن وزن‌دهی مناسب در زمان
$t-1$
میان متخصص‌ها است که به تخمین زمان سفر در زمان
$t$
نزدیک باشد.

\textbf{
ویژگی‌های مسئله
}:
\begin{itemize}
\item\textbf{
ابعاد نمایی:
}
همانگونه از تعریف متخصص‌ها (مسیرهای میان مبداء و مقصد) بر می‌آید، تعداد این مسیرها با ابعاد گراف به شکل نمایی زیاد می‌شود.

\item\textbf{
تغییر با زمان:
}
همان‌گونه که از ترافیک در شهر انتظار می‌رود، کوتاه ترین مسیر و زمان سفر میان یک مبداء و مقصد ثابت، با گذشت زمان تغییر می‌کند.
\item\textbf{
زمان محاسبه‌ی تخمین:
}
یک چالش دیگر، زمانی است که به طول می‌انجامد تا تخمینی از زمان سفر ارائه شود (یک رقیب جدی الگوریتم دایجسترا
\LTRfootnote{
Dijkstra's Algorithm
}
برای یافتن کوتاه‌ترین مسیر در گراف وزن‌دار است).
\end{itemize}


% --> Stock Prediction
\subsubsection{
پیش‌بینی رفتار سهام
}


قیمت یک سهم در بازار سهام، یک سری زمانی است که با توجه به عدم دانش ما، یک متغیر تصادفی می‌تواند فرض شود. با چشم پوشی از فرض‌های مختلفی که در رابطه با توزیع این متغیر تصادفی می‌توان داشت، به بررسی استفاده از نظر متخصص‌ها برای پیش‌بینی رفتار این سری زمانی می‌خواهیم بپردازیم. به بیان ساده، در هر زمان، تعدادی نظر از متخصص‌های مختلف در رابطه با رفتار لحظه‌ی بعد قیمت یک سهم دریافت می‌کنیم و می‌خواهیم با استفاده از این نظرها، به پیش‌بینی بپردازیم.

سهم مورد نظر را به صورت دنباله‌ی
$\{ y_t \}_{t \in \hollow{N}} \in \set{Y}$
در نظر بگیرید که در آن مجموعه‌ی پیش‌آمدهای مختلف
$\set{Y}$
و اندیس
$t$
زمان را نشان می‌دهد. مجموعه‌ای از متخصص‌ها
$\set{E} = \left\lbrace \mthfnc{e}_i \vert \mthfnc{e}_i: \hollow{N} \rightarrow \set{Y} \right\rbrace^{k}_{i=1}$
را در نظر بگیرید که در مقدار
$\mthfnc{e}_i(t)$
نشان‌دهنده‌ی پیش‌بینی متخصص
$i$
ام برای رفتار قیمت سهم در زمان
$t+1$
است. ما به دنبال روشی برای وزن‌دهی (اعتماد) به متخصص‌ها و تصمیم‌گیری هستیم که با استفاده از آن، پیش‌بینی با خطای کمتری برای رفتار سهم بدست آوریم.

اگر فرض کنیم که فضای پیش‌آمدها محدب است می‌توانیم به مسئله به شکل:
\[
\mbox{find } x_{t-1} \in \simplex{k} \mbox{ s.t. } \sum^{k}_{i=1} x_{t-1}(i) \mthfnc{e}_i(t-1) \approx y_t
\]
مسئله به بیان یادگیری به کمک توصیه‌ی متخصص‌ها: هدف یافتن وزن‌دهی مناسب به متخصص‌ها است به گونه‌ای که رفتار سهم را بخوبی بتوانیم در لحظه‌های بعد پیش‌بینی کنیم.

\textbf{
ویژگی‌های مسئله:
}
\begin{itemize}
\item\textbf{
معیار سنجش نامعلوم:
}
معیاری که به طور معمول از نظریه‌ی بازی
\LTRfootnote{
Game Theory
}
می‌آید، پشیمانی
\LTRfootnote{
Regret
}
است. به این معنی که میزان اشتباه روش را با یک دسته‌ی از پیش مشخص شده از متخصص‌ها یا روش‌ها مقایسه می‌کند. ابن معیار در شرایطی که هیچ کدام از متخصص‌ها خطای کلّی کوچکی نداشته باشد، ناکارآمد است.
\item\textbf{
تغییر با زمان:
}
رفتار قیمت یک سهم، با توجه به عوامل بیرونی (که در مدل بالا در نظر گرفته نشده‌اند)، در زمان تغییر می‌کند. این به معنای این است که عملکرد متخصص‌ها در بازه‌های مختلف دستخوش تغییر می‌شود. به همین منظور باید به گونه‌ای مسئله را مورد بررسی قرار داد که بتوان نسبت به تغییرات نیز انعطاف پدیر بود.
\end{itemize}




% ---------------------------------------
% --> Background
\subsection{
پیش‌نیازها
}

پیش‌نیازهای نظری و عملی خواندن این پایان‌نامه اینجا نوشته می‌شود.


فهرست پیش‌نیازهایی که لازم به نوشتن هستن (به‌روز می‌شود):
\begin{itemize}
\item {\bf
نمادگزاری زمان اجرای الگوریتم‌ها
}


\item {\bf
بهینه‌سازی محدب
\LTRfootnote{
Convex Optimization
}
}

\item {\bf
بهینه‌سازی برخط
\LTRfootnote{
Online Optimization
}
}

\item {\bf
رانش مفهوم
\LTRfootnote{
Concept Drift
}
}

\end{itemize}







% --> Computational Complexity Notations
% !TEX encoding = UTF-8 Unicode
% ---------------------------------------
% Reference Parameters
% Author: T. H. Cormen, C. E. Leiserson, R. L. Rivest, C. Stein
% Title: Introduction to Algorithms
% Publisher: The MIT Press
% Date: 2009

% ---------------------------------------
% TODO
%
%
%
%
%
%
%
%
%
%
%
%

% ---------------------------------------
% Computational Complexity Notations
\subsection{
نمادگزاری زمان اجرای الگوریتم
\cite{clrs2009}
}
برای تحلیل و بررسی رفتار تابع‌ها و الگوریتم‌های مختلف از
\textit{
نمادگذاری مجانبی
}\LTRfootnote{
Asymptotic Notation
}
استفاده می‌کنیم. تحلیل‌های
\textit{
غیر مجانبی
}\LTRfootnote{
Non-Asymptotic
}
نیز برای تحلیل الگوریتم‌ها و تحلیل رفتار تابع‌ها وجود دارد که در این پایان‌نامه بررسی نمی‌شود.

% ---------------------------------------

توجه کنید که در این بخش تنها به تعریف نمادگذاری و تحلیل رفتار مجانبی تابع پرداخته می‌شود. در رابطه با اینکه هر الگوریتم چگونه باید تحلیل شود، پس از ارائه‌ی بحث می‌شود. به عنوان مثال این موضوع که رفتار یک الگوریتم باید در قالب تعداد
\textit{
عملیات‌های ممیزشناور
}\LTRfootnote{
Floating Point Operations (FLOPs)
}
بیان شود و یا بر اساس تعداد
\textit{
گام‌ها
}\LTRfootnote{
Iterations
}
و یا ،در برخی موارد، بر حسب تعداد
\textit{
جستارها به یک اوراکل
}\LTRfootnote{
Queries to Oracle
}.
در ادبیات رایج است که در رابطه با تعداد گام‌ها صحبت شود و رفتار مجانبی بر حسب این عدد تحلیل شود. هرچند در کاربردها، باید دقت شود که کدام‌یک از تحلیل‌ها مرتبط خواهد بود زیرا عملکرد کلی الگوریتم به صورت مجموعه‌ای از تحلیل‌ها توصیف می‌شود. به عنوان نمونه اگر بتوان هزینه (زمان) هر جستار به یک اوراکل را ارزان (کم) کرد، می‌توان از روش‌هایی استفاده کرد که از نظر تعداد جستارها به یک اوراکل رفتار بدتری دارند و ممکن است عملکرد زمانی بهتری دریافت کرد.

% ---------------------------------------

به عنوان کلام آخر در این موضوع، توجه داشته باشید که عملکرد زمانی نیازمند تحلیل جداگانه‌ای است با توجه به زیرساخت‌هایی است که الگوریتم‌ها بر آن اجرا می‌شوند. این موضوع می‌تواند باعث شود که به‌روزترین روش‌ها که از یک منظر رفتار مجانبی بسیار شگفت‌انگیزی دارند در مسئله‌ی عملی شما کارا نباشند.


\subsubsection{
نمادگذاری مجانبی
}
نمادگذاری استفاده شده برای توصیف رفتار مجانبی الگوریتم‌ها، به شکل تابع‌هایی با دامنه و برد در اعداد طبیعی است. این نوع نمادگذاری‌ها برای بررسی
\textit{
بدترین حالت
}\LTRfootnote{
Worst-case
}
زمان اجرای یک تابع
$\mthfnc{T}(n)$
مناسب است. گاهی مناسب است که از نمادگذاری سوءاستفاده کنیم. به طور مثال، ممکن است دامنه‌ی تابع را اعداد حقیقی و یا زیرمجموعه‌ای از اعداد صحیح بگیریم. اما همواره باید مطمئن شد که منظور دقیق از نمادگذاری چیست که اگر از نمادگذاری سوءاستفاده شد، به استفاده‌ی اشتباه منجر نشود.

% ---------------------------------------

\textbf{
نمادگذاری
$\Theta$
}\\
برای تابع داده‌ی شده‌ی
$\mthfnc{g}: \hollow{N} \rightarrow \hollow{N}$،
مجموعه‌ی
$\Theta( \mthfnc{g} )$
را به شکل زیر تعریف می‌کنیم:
\[
\Theta( \mthfnc{g} ) = \left\lbrace \mthfnc{f}: \hollow{N} \rightarrow \hollow{N} \;\vert\; \exists c_l,c_u \geq 0, \; \exists n_0 \in \hollow{N} \; : \; \forall n \geq n_0 0 \leq c_l \mthfnc{g}(n) \leq \mthfnc{f}(n) \leq c_u \mthfnc{g}(n) \right\rbrace
\]
به صورت کلامی، تابع
$\mthfnc{f}$
به
$\Theta( \mthfnc{g} )$
تعلق دارد اگر بتواند در مقدارهای بسیار بزرگ، تابع
$\mthfnc{g}$
را با دو تابع
$c_l \mthfnc{g}$
و
$c_u \mthfnc{g}$
از پایین و بالا کراندار کرد. در این صورت می‌نویسیم
$\mthfnc{f} \in \Theta( \mthfnc{g} )$
و یا با سوءاستفاده از نمادگذاری می‌نویسیم
$\mthfnc{f} = \Theta( \mthfnc{g} )$.
و این رابطه را می‌خوانیم:
\textit{
$\mthfnc{g}$
یک کران مجانبا تنگ
\LTRfootnote{
Asymptotically Tight Bound
}
 برای
$\mthfnc{f}$
}
است. توجه کنید که تمامی اعضای مجموعه‌ی
$\Theta( \mthfnc{g} )$
باید
\textit{
مجانبا نامنفی
}\LTRfootnote{
Asymptotically Non-negative
}
باشند. از این به بعد در این قسمت فرض می‌کنیم که تمامی توابع مجانبا نامنفی هستند (که در تحلیل پیچیدگی محاسباتی
\LTRfootnote{
Computational Complexity
}
و زمان اجرا، فرض محدود کننده‌ای نیست).

% ---------------------------------------

\subsubsection{
نمادگذاری
$O$
}
نمادگذاری
$\Theta$
یک تابع را مجانبا از بالا و پایین کران‌دار می‌کرد. زمانی که تنها یک کران مجانبی از بالا داشته باشیم، از نمادگذاری
$O$
استفاده می‌کنیم.
\[
O( \mthfnc{g} ) = \left\lbrace \mthfnc{f}: \hollow{N} \rightarrow \hollow{N} \;\vert\; \exists c_u \geq 0, \; \exists n_0 \in \hollow{N} \; : \; \forall n \geq n_0 \; 0 \leq \mthfnc{f}(n) \leq c_u \mthfnc{g}(n) \right\rbrace
\]
از نمادگذاری
$O$
برای معرفی کران‌بالا برای یک تابع استفاده می‌شود، با دقت یک ضریب ثابت. به این معنی که در مقدارهای بزرگ‌تر از
$n_0$،
مقدار تابع
$\mthfnc{f}$
توسط ضریبی مثبت از تابع
$\mthfnc{g}$
از بالا کراندار می‌شود.

% ---------------------------------------

توجه کنید که با توجه به تعریف داریم
$\Theta( \mthfnc{g} ) \subseteq O( \mthfnc{g} )$
است زیرا که نمادگذاری
$\Theta$
کران‌دار کردن مجانبی از بالا و پایین را نتیجه می‌دهد.

% ---------------------------------------

دقت کنید که در بعضی متون در ادبیات بررسی پیچیدگی تابع‌ها، از نماد
$O$
به عنوان یک کران بالای مجانبا تنگ استفاده می‌شود در حالی که نماد انتخاب شده در این پایان‌نامه برای نمایش کران‌های مجانبا تنگ
$\Theta$
است. پس هرگاه از نماد
$O$
استفاده می‌شود، کران بالای مجانبی منظور است و در مورد تنگ بودن این کران فرضی وجود ندارد. این تمایز میان کران بالای مجانبی و کران بالای مجنابی تنگ در ادبیات پیچیدگی محاسبات کاری استاندارد است.

% ---------------------------------------

با استفاده از نمادگذاری
$O$
معمولا می‌توان زمان اجرای الگوریتم‌ها را با نگاه به ساختار کلّی آن توصیف کرد. به طور مثال:
\begin{algorithm}[H]
\caption*{
حلقه‌ی ساده
}
\begin{algorithmic}[1]
\REQUIRE
مسئله با بعد
$n$
  \FOR{ $i = 1$
  تا
  $N$ }
  \FOR{ $j = 1$
  تا
  $M$ }
  \STATE
  انجام یک سری عملیات
  \ENDFOR
  \ENDFOR
\end{algorithmic}
\end{algorithm}
فرض اینکه
\textit{
انجام یک سری عملیات
}
به طوری مجانبی از بالا توسط
$O( n )$
کراندار می‌شود، یک کران مجانبی بالا برای تعداد عملیات‌های این الگوریتم
$O( N  M n )$
خواهد بود. توجه کنید که این کران تنها به تحلیل یک گام در حلقه بستگی دارد و اگر به طور مثال کران بهتری برای بدنه‌ی حلقه بتوان یافت آنگاه کران کلّی بهتری می‌توان ارائه کرد و این ناشی از این است که نمادگذاری
$O$
تنها یک کران بالا را نشان می‌دهد که لزوما تنگ نیست.

% ---------------------------------------

خوب است باز اشاره کنیم که تحلیل الگوریتم‌ها معمولا بر اساس بدترین حالت و یا متوسط زمان اجرا
\LTRfootnote{
Expected running-time
}
است. در صورتی که به بیان بدترین حالت پیچیدگی الگوریتم تحلیل شود، می‌توان انتظار داشت که اجراهایی با زمان کمتر نیز وجود داشته باشند (در برخی مسئله‌های عملی، رخ دادن بدترین حالت کاملا نادر و در برخی کاملا رایج است). در صورتی که تحلیل بر اساس متوسط زمان اجرا باشد، اجراهایی با زمان کم‌تر و بیش‌تر را می‌توان انتظار داشت که فرکانس رخ‌دادن آن‌ها بستگی به توزیع ورودی مسئله دارد.

% ---------------------------------------

\subsubsection{
نمادگذاری
$\Omega$
}
همانگونه که نمادگذاری
$O$
در رابطه با کران مجانبی بالا صحبت می‌کرد، نمادگذاری
$\Omega$
در رابطه با یک کران مجانبی پایین صحبت می‌کند.

\[
\Omega( \mthfnc{g} ) = \left\lbrace \mthfnc{f}: \hollow{N} \rightarrow \hollow{N} \;\vert\; \exists c_l \geq 0, \; \exists n_0 \in \hollow{N} \; : \; \forall n \geq n_0 \; 0 \leq c_u \mthfnc{g}(n) \leq \mthfnc{f}(n) \right\rbrace
\]

طبق قضیه‌ی ۳.۱
\cite{clrs2009}
به ازای هر تابع
$f$
داریم
\[
\Theta( \mthfnc{f} ) = O( \mthfnc{f} ) \cap \Omega( \mthfnc{f} )
\]

% ---------------------------------------

زمانی که می‌گوییم
\begin{quote}
زمان اجرای الگوریتم
$\Omega( \mthfnc{g}(n) )$
است، منظور این است که به طور مجانبی، در سریع‌ترین حالت نیز به اندازه‌ی ضریبی ثابت از
$\mthfnc{g}(n)$
محاسبه نیاز است. یا به طور معادل یک ضریب ثابتی از
$\mthfnc{g}(n)$
کران مجانبی پایین برای سریع‌ترین اجرای مسئله است.
\end{quote}

% ---------------------------------------

حال به جمع‌بندی و بیانی دیگر از رفتار تابع‌ها می‌پردازیم. تابع‌های
$\mthfnc{f} : \hollow{N} \rightarrow \hollow{N}$
و
$\mthfnc{g} : \hollow{N} \rightarrow \hollow{N}$
را در نظر بگیرید. حالت‌های زیر رابطه‌ی مختلف میان رفتار آن‌ها را نشان می‌دهد:
\begin{itemize}
\item $\lim_{n\rightarrow \infty} \frac{ \mthfnc{f}(n) }{ \mthfnc{g}(n) } = \infty$\\
آنگاه رشد مجانبی
$\mthfnc{f}$
بیشتر از
$\mthfnc{g}$
است و داریم:
$\mthfnc{f} \in O( \mthfnc{g} )$
و
$\mthfnc{g} \in \Omega ( \mthfnc{f} ) $

\item $\lim_{n\rightarrow \infty} \frac{ \mthfnc{f}(n) }{ \mthfnc{g}(n) } = C > 0$\\
آنگاه رشد مجانبی
$\mthfnc{f}$
و
$\mthfnc{g}$
از یک مرتبه است و داریم:
$\mthfnc{f} \in \Theta( \mthfnc{g} )$
و
$\mthfnc{g} \in \Theta( \mthfnc{f} )$


\end{itemize}



% --> Convex Optimization
% ---------------------------------------
% Reference Parameters
% Authors: Stephen Boyd, Lieven Vandenberghe
% Title: Convex Optimization
% Publisher: Cambride University Press
% Date: 2004

% ---------------------------------------
% Online Optimization
\subsubsection{
بهینه‌سازی محدب
\cite{convexoptimization}
}


% ---------------------------------------
% TODO
\begin{itemize}
\item
۱. مقدمه
\item
۲. مجموعه‌های محدب (کلّیت)
\item
۳. تابع‌های محدب (کلّیت)
\item
۴. مسئله‌های بهینه‌سازی محدب (کلّیت)
\item
۶. تخمین و فیت کردن (بخون)
\item
۷. تخمین آماری (بخون)
\item
۹. بهینه‌سازی بدون قید (الگوریتم‌ها و نرخ‌های همگرایی)
\item
۱۱. روش نقطه‌ی درونی (بخون)

\end{itemize}


% ---------------------------------------
% Mathemtical Optimization
\subsubsubsection{
بهینه‌سازی ریاضی
\LTRfootnote{
Mathematical Optimization
}
}
یک مسئله‌ی بهینه‌سازی ریاضی یا به عبارت ساده‌تر 
\emph{
مسئله‌ی بهینه‌سازی
\LTRfootnote{
Optimization Problem
}
} 
به شکل زیر است:
\[
\begin{array}{lll}
\mbox{minimize} & \mthfnc{f}_0(x) & \\
\mbox{subject to} & \mthfnc{f}_i(x) \leq b_i, & i = 1, \cdots, m
\end{array}
\]
که در آن 
$x = (x_1, \cdots, x_n)$ 
متغیرهای بهینه‌سازی مسئله، تابع 
$\mthfnc{f}_0: \hollow{R}^n \rightarrow \hollow{R}$ 
تابع هدف
\LTRfootnote{
Objective Function
} 
و تابع‌های 
$\mthfnc{f}_i: \hollow{R}^n \rightarrow \hollow{R}$ 
تابع‌های قید
\LTRfootnote{
Constraint Functions
} 
هستند. بردار 
$x^*$ 
را بهینه
\LTRfootnote{
Optimal
} 
یا پاسخ مسئله گویند اگر در میان بردارهایی که در قید‌ها صدق می‌کنند، کمترین مقدار تابع هدف را داشته باشد.
\[
\forall z \in \hollow{R}^n: \mthfnc{f}_1(z) \leq b_1 , \cdots, \mthfnc{f}_m(z) \leq b_m \Rightarrow \mthfnc{f}_0(z) \geq \mthfnc{f}_0(x^*)
\]

به یک مسئله‌ی بهینه‌سازی، برنامه‌ریزی خطی
\LTRfootnote{
Linear Program
} 
یا بهینه‌سازی خطی
\LTRfootnote{
Linear Optimization
} 
گویند هرگاه (تابع هدف و تابع‌های قید همگی خطی باشند):
\[
\forall \alpha,\beta \in \hollow{R} \; : \; \mthfnc{f}_i( \alpha x + \beta y ) = \alpha \mthfnc{f}_i(x) + \beta \mthfnc{f}_i(y) \hspace{5mm} i = 0,1,\cdots,m
\]

هرگاه تابع هدف و تابع‌های قید همگی در ویژگی زیر صدق کنند (محدب
\LTRfootnote{
Convex
} 
باشند) مسئله را بهینه‌سازی محدب
\LTRfootnote{
Convext Optimization
} 
می‌نامند:
\[
\forall \alpha,\beta \in [0,1] \; , \; \alpha + \beta = 1 \; : \; \mthfnc{f}_i( \alpha x + \beta y ) \leq \alpha \mthfnc{f}_i(x) + \beta \mthfnc{f}_i(y) \hspace{5mm} i = 0,1,\cdots,m
\]


% ---------------------------------------
% Applications
\subsubsubsection{
کاربردها
\LTRfootnote{
Applications
}
}

مسئله‌ی بهینه‌سازی ***، مسئله‌ی انتخاب 
\textbf{
بهترین
} 
بردار 
$x^* \in \hollow{R}^n$ 
از میان بردارهایی است که در شرایط 
$\mthfnc{f}_i(x) \leq b_i$ 
صدق می‌کنند. در جمله‌ی پیش، 
\textbf{
بهترین
} 
کیفیتی است که به وسیله‌ی تابع 
$\mthfnc{f}_0: \hollow{R}^n \rightarrow \hollow{R}$ 
کمّی شده است به این معنی که به با ثابت نگاه داشتن تابع‌های قید (نقاطی که در میان آن‌ها به دنبال پاسخ هستیم)، می‌توان کمّی سازی‌های متفاوت و احتمالا نتایج متفاوتی برای 
\emph{
بهترین
} 
بردار بدست آورد.

چند مثال از مسئله‌هایی که در چهارچوب مسائل بهینه‌سازی قرار می‌گیرند.

% Portfolio Optimization
\subsubsubsection{
بهینه‌سازی پورتفولیو
\LTRfootnote{
Portfolio Optimization
}:
} 
در مسئله‌ی بهینه‌سازی پورتفولیو، ما به دنبال سرمایه‌گذاری سرمایه‌ی
\LTRfootnote{
Capital
} 
خود در چند دارایی
\LTRfootnote{
Asset
} 
هستیم. متغیر 
$x_i$ 
نشان دهنده‌ی سرمایه‌گذاری در دارایی 
$i$
ام است و بردار 
$x \in \hollow{R}^n$ 
نشان دهنده‌ی اختصاص پورتفولیو به دارایی‌های مختلف است. قیدها به طور مثال می‌توانند محدودیت‌هایی بر سرمایه، منفی نبودن سرمایه‌ی اختصاص داده شده و حداقل سود بردار سرمایه‌گذاری باشند. هدف می‌تواند کم کردن مقدار ریسک
\LTRfootnote{
Risk
} 
پورتفولیو باشد.


% Stock
\subsubsubsection{
پیش‌بینی سهام (آفلاین)
\LTRfootnote{
(Offline) Stock Prediction
}
} 
تابع‌های 
$\mthfnc{e}_i : \hollow{R}^d \rightarrow \hollow{R} \; i=1,\cdots,k$، 
و جفت نقطه‌های 
$\{ (x_j, y_j) \}^{N}_{j=1}$ 
داده شده‌اند. هدف یافتن ضریب‌های 
$\alpha = (\alpha_i) \in \hollow{R}^k \; \sum^{k}_{i=1} \alpha_i = 1$ 
است به گونه‌ای که تابع هزینه‌ی داده شده 
$\mthfnc{l} : \hollow{R} \rightarrow \hollow{R}$ 
کمینه شود. ابتدا تابع پیش‌بینی
\[
\mthfnc{f}(\alpha, x) = \sum^{k}_{i=1} \alpha_i \mthfnc{e}_i(x)
\]
و همچنین تابع هدف تجمعی 
\[
\mthfnc{L}(\alpha) = \sum^{N}_{j=1} \mthfnc{l} ( \mthfnc{f}(\alpha, x_j) , y_j )
\]
حال مسئله‌ی بهینه‌سازی را به صورتی که از پیش دیده‌ایم باز نویسی می‌کنیم:
\[
\begin{array}{lll}
\mbox{minimize} & \mthfnc{L}(\alpha) & \\
\mbox{subject to} & \alpha_i \geq 0 & i = 1,\cdots, k\\
& \sum^{k}_{i=1} \alpha_i = 1 & 
\end{array}
\]

در واقع در این مسئله، اطلاعات 
$k$ 
متخصص
\LTRfootnote{
Expert
} 
داده شده است و هدف ما این است که با توجه به سابقه‌ای که از عملکرد در داده‌های 
$\{ (x_j, y_j) \}^{N}_{j=1}$ 
دیده می‌شود، تصمیم بگیریم که چقدر به هریک از متخصص‌ها اعتماد کنیم (به پیش‌بینی او، وزن نسبت دهیم).

در نسخه‌ی ارائه شده از این مسئله، تمامی داده‌ها همزمان دانسته فرض شده‌اند و ما به دنبال کمینه کردن خطای پیش‌بینی در آن‌ها هستیم. این مسئله هم بدون در نظر گرفتن منظم‌سازی
\LTRfootnote{
Regularization
} 
صورت‌بندی شده است و هم بدون در نظر گرفتن دریافت برخط
\LTRfootnote{
Online
} 
داده‌ها. در بخش‌های آینده به این دو ویژگی پرداخته خواهد شد.

% Traffic
\subsubsubsection{
تخمین مسیر / زمان سفر (آفلاین)
\LTRfootnote{
Route Finding \ ETA Estimation
}
}
نقشه‌ی یک شهر را به شکل یک گراف 
$G = (\set{V},\set{E})$ 
در نظر بگیرید که در آن 
$\set{V}$ 
مجموعه‌ی تقاطع‌ها و نقاطع تولید سفر، و 
$\set{E}$ 
مجموعه‌ی خیابان‌ها و مسیرها است. یک جفت نقطه‌ی متمایز را بر روی نقشه انتخاب کنید، 
$(src, dst) \in \set{V} \times \set{V}$ 
مجموعه‌ی مسیرهای ساده
\LTRfootnote{
Simple Path
} 
(بدون دور) شروع شده از 
$src$ 
و تمام شده در 
$dst$ 
را در نظر بگیرید:
\[
\set{P}(src,dst) = \{ \mbox{all paths from } src \mbox{ to } dst \} = \{ p_i \}^{k}_{i=1}
\]
می‌خواهیم با داشتن زمان سفر متوسط هر یک از یال‌ها (که همان تابع توزیع ترافیک است) در چند روز مختلف،
\[
\mthfnc{t}_j : \set{E} \rightarrow \hollow{R}^+ \; j = 1,\cdots,N
\] 
زمان و مسیر کوتاه ترین سفر میان مبداء و مقصد تعیین شده را تخمین بزنیم. در این مسئله به دنبال توزیع وزن میان مسیرهای مختلف بین مبداء و مقصد هستیم. یعنی به دنبال نقطه‌ی 
$\alpha \in [0,1]^k$ 
به گونه‌ای که 
$\sum^{k}_{i=1} \alpha_i = 1$. 
تابع خطای جزئی و تجمعی نیز به شکل زیر تعریف می‌شوند:
\[
\begin{split}
&\mthfnc{l}(\alpha, \mthfnc{t}_j) = \sum^{k}_{i=1} \alpha_i \sum_{e \in p_i} \mthfnc{t}_j(e)\\
&\mthfnc{L}(\alpha) = \sum^{N}_{j=1} \mthfnc{l}( \alpha, \mthfnc{t}_j )
\end{split}
\]
صورت بندی به شکل مسئله‌ی بهینه‌سازی در زیر آمده است:
\[
\begin{array}{lll}
\mbox{minimize} & \mthfnc{L}(\alpha) & \\
\mbox{subject to} & \alpha_i \geq 0 & i = 1,\cdots, k\\
& \sum^{k}_{i=1} \alpha_i = 1 & 
\end{array}
\]

در اینجا نیز مسئله به این صورت قابل توصیف است که میان دو نقطه‌ی مبداء و مقصد 
$(src, dst)$ 
تعداد 
$k$ 
مسیر داده شده است. هدف تخصیص وزن به مسیرهای مختلف است به گونه‌ای که متوسط زمان سفر در مشاهده‌های ترافیک گراف 
$\{ \mthfnc{t}_j \}^{N}_{j=1}$ 
در مسیرهای وزن‌دار، کمینه شود. زیاد بودن وزن مسیر 
$p_i$ 
نسبت به مسیر 
$p_{i^\prime}$، 
نمایانگر کوتاه‌تر بودن زمان سفر در تاریخچه‌ی مشاهده شده در مسیر 
$p_i$ 
نسبت به 
$p_{i^\prime}$ 
با توجه به تابع هزینه‌ی 
$\mthfnc{l}$ 
است.





% ---------------------------------------
% Solving Optimization Problems
\subsubsubsection{
حل کردن مسئله‌های بهینه‌سازی
}
یک روش حل
\LTRfootnote{
Solution Method
} 
برای دسته‌ای از مسئله‌های بهینه‌سازی، یک الگوریتم
\LTRfootnote{
Algorithm
} است که جواب بهینه را با داده شدن یک نمونه از مسئله‌
\LTRfootnote{
Instance of Problem
} 
محاسبه می‌کند. از اواخر دهه‌ی ۱۹۴۰، تلاش‌های زیادی در راستای توسعه‌ی الگوریتم‌هایی برای حل دسته‌های مختلفی از مسئله‌های بهینه‌سازی، تحلیل ویژگی‌های آن‌ها و توسعه‌ی پیاده‌سازی‌های خوب نرم‌افزاری از آنها، شده است. موثر بودن این الگوریتم‌ها، به طور مثال توانایی آن‌ها در حل مسئله‌ی بهینه‌سازی ***، تفاوت‌های مشهود و وابسته به عواملی همچون شکل تابع‌های هدف و قید، تعداد متغیرها و قیدها، و ساختارهای خاص (به طور مثال تنکی
\LTRfootnote{
Sparsity
}) 
دارد.

%%%%%%%%%%%

حتی زمانی که تابع هدف و تابع‌های قید هموار
\LTRfootnote{
Smooth
} 
باشند (به طور مثال چند جمله‌ای‌ها)، حل مسئله‌ی بهینه‌سازی *** به طرز شگفت‌انگیزی دشوار است. به همین دلیل رویکردهای بررسی مسئله‌ی کلی، دارای برخی سازش‌ها
\LTRfootnote{
Compromise
} 
همچون زمان بسیار طولانی محاسبه و یا احتمال پیدا نکردن جواب، هستند.

%%%%%%%%%%%

اما در این میان، برخی استثناء‌ها نیز وجود دارند. برای برخی دسته از مسئله‌ها، الگوریتم‌های موثری وجود دارد که جواب را با اطمینان می‌یابند (حتی برای مسئله‌هایی با صدها یا هزارها متغیر و قید). از جمله‌ی این کلاس‌ها، می‌توان به برنامه‌ریزی خطی
\LTRfootnote{
Linear Programming
}، 
مسئله‌های کمینه‌ی مربعات
\LTRfootnote{
Least-Squares Problems
} 
و بهینه‌سازی محدب اشاره کرد.


% ---------------------------------------
% Nonlinear Optimization
\subsubsubsection{
بهینه‌سازی غیرخطی
}






































% --> Intorduction to Online Optimization
% ---------------------------------------
% Reference Parameters
% Author: Sebastien Bubeck
% Title: Introduction to Online Optimization
% Publisher: Princeton University - Department of Operations Research and Financial Engineering
% Date: December 14, 2011

% ---------------------------------------
% TODO
% 
%
%
%
%
%
%
%
%
%
%
%

% ---------------------------------------
% Online Optimization
\subsubsection{
بهینه‌سازی برخط
\cite{onlinelearningbubeck}
}

% ---------------------------------------
% Statistical Learning Theory
\subsubsubsection{
نظریه‌ی یادگیری آماری
\LTRfootnote{
Statistical Learning Theory
}
}

در دنیایی که جمع‌آوری خودکار داده فراگیر شده است، آماردان‌ها باید چهارچوب‌های فکری خودشان را با مسئله‌های جدید مطابقت دهند. هر زمان که صحبت از شبکه‌ی اینترنت، داده‌های مشتری
\LTRfootnote{
Consumer Data
}
، و یا بازار مالی
\LTRfootnote{
Financial Market
}
 با میان می‌آید، یک ویژگی مشترک به چشم می‌آید: حجم بسیار زیاد داده که نیازمند فهم و تحلیل سریع است. شرایط مسئله‌های کنونی با آمار کلاسیک بسیار تفاوت دارد، تعداد مشاهده‌ها بسیار زیاد و تعداد متغیرهای مرتبط بسیار کم است. یکی از مدل‌های به تفصیل بررسی شده برای یادگیری، چهارچوب نظریه‌ی یادگیری آماری است. که به اختصار در ادامه بررسی خواهد شد:

% Definition
\begin{quote}
{\bf
پروتکل
\LTRfootnote{
Protocol
}
 یادگیری آماری:
} 
پروتکل ابتدایی یادگیری آماری به شکل زیر است:
\begin{itemize}
\item
مشاهد کنید نمونه‌های 
$Z_1, \cdots, Z_n \in \randomvariable{Z}$. 
فرض می‌کنیم که توزیع مشاهده‌ها مستقل و هم‌توزیع
\LTRfootnote{
I.I.D.
}
 از توزیع احتمال 
$\hollow{P}$
آمده اند.

\item
تصمیم بگیرید (یا عملی را انتخاب کنید) 
$a( Z_1, \cdots, Z_n ) \in \set{A}$ 
که 
$\set{A}$ 
یک مجموعه‌ی داده شده از تصمیم‌های ممکن است.

\item
به اندازه‌ی متوسط ضرر
\LTRfootnote{
Loss
} 
$\hollow{E}_{Z \sim \hollow{P}} \mthfnc{l}( a(Z_1,\cdots,Z_n), Z )$ 
هزینه بپردازید، که در آن 
$\mthfnc{l} : \set{A} \times \randomvariable{Z} \rightarrow \hollow{R}_+$ 
تابع ضرر داده شده است.
\end{itemize}

{\bf
هدف:
} 
کمینه کردن (و کنترل) ریسک مازاد
\LTRfootnote{
Excess Risk
}
:
\[
r_n = \hollow{E}_{Z \sim \hollow{P}} \mthfnc{l}( a(Z_1, \cdots, Z_n), Z ) - \inf_{a \in \set{A}} \hollow{E}_{Z \sim \hollow{P}} \mthfnc{l} ( a, Z )
\]
که معیار اندازه‌گیری میزان متوسط ضرر متحمل شده در مقایسه با انتخاب بهینه است.

\end{quote}



% Remark
\begin{quote}
{\bf
تعریف:
} 
کنترل کردن ریسک مازاد، به معنی پیدا کردن کران بالا برای 
$r_n$ 
است که یا در امید ریاضی
\LTRfootnote{
Expectation
} 
صدق کند ( نسبت به دنباله‌ی 
$Z_1,\cdots,Z_n$ 
) و یا با احتمال حداقل 
$1 - \delta$. 
معمولا کران بالا به شکلی عبارتی است بر اساس معیارهایی از پیچیدگی
\LTRfootnote{
Complexity Measure
} 
$\set{A}$ 
و 
$\mthfnc{l}$. 
همچنین اگر کران بالا به توزیع 
$\hollow{P}$ 
بستگی داشته باشد، می‌گوییم کران بالا وابسته به توزیع
\LTRfootnote{
Distribution-Dependent
} 
است و در غیر این صورت گوییم که آزاد از توزیع
\LTRfootnote{
Distribution-Free
} 
است.

\end{quote}


فرمول‌بندی بالا کلّی بوده و باعث می‌شود که مسئله‌های بسیاری را در بر گیرد. در ادامه به چند مثال اشاره می‌کنیم:

% Example
{\bf
مثال: تخمین رگرسیون
\LTRfootnote{
Regression Estimation
}
} 
\begin{itemize}
\item
داده‌ی مشاهده شده جفت‌هایی از نقطه‌ها به شکل 
$ Z_i = ( X_i, Y_i ) \in \set{X} \times \set{Y} $ 
هستند.


\item
مجموعه‌ی ممکن انتخاب‌ها، مجموعه‌ی تابع‌های از 
$\set{X}$ 
به 
$\set{Y}$ 
است، 
$\set{A} \subset \{ \mthfnc{f} : \set{X} \rightarrow \set{Y} \} $.


\item
تابع هزینه 
$\mthfnc{l}(a, (x,y) )$ 
میزان دقت پیش‌بینی تابع 
$\mthfnc{a}: \set{X} \rightarrow \set{Y}$ 
را اندازه‌گیری می‌کند. به عنوان نمونه اگر 
$\set{Y}$ 
یک فضای نرم‌دار
\LTRfootnote{
Normed Space
}
 باشد، یک انتخاب معمول 
$\mthfnc{l}(a, (x,y)) = || a(x) - y ||$ 
خواهد بود.

\end{itemize}

% Example
{\bf
مثال: تکمیل ماتریس
\LTRfootnote{
Matrix Completion
}
 (یا فیلتر کردن مبتنی بر همکاری
\LTRfootnote{
Collaborative Filtering
}
 )
} 
تکمیل ماتریس یک مثال از یادگیری در ابعاد بالا
\LTRfootnote{
High-Dimensional Learning
} 
است. در این مسئله 
$\hollow{P}$ 
یک توزیع یکنواخت بر روی درایه‌های مجهول ماتریس 
$ M \in \hollow{R}^{m \times d }$، 
و هدف بازسازی ماتریس 
$M$ 
است. سازوکار ابعاد بالا مربوط می‌شود به شرایط 
$ n \ll m \times d$ 
برقرار باشد. برای عملی
\LTRfootnote{
Feasible
} 
کردن مسئله، یک فرض طبیعی این است که ماتریس 
$M$ 
رتبه‌ی
\LTRfootnote{
Rank
} 
کمی دارد در مقایسه با تعداد نمونه‌ها 
$n$، 
بنابراین خواهیم داشت 
$k < n \ll m \times d$. 
تابع ضررهای مختلفی می‌توان در نظر گرفت، چه در مسئله‌ی پیش‌بینی و چه در مسئله‌ی تخمین در کامل‌سازی ماتریس.




% ---------------------------------------
% Statistical Learning Theory
\subsubsubsection{
یادگیری برخط
\LTRfootnote{
Online Learning
}
}

با وجود موفقیت‌های بسیاری که نظریه‌ی یادگیری آماری بدست آورده، در زمینه‌ی پویایی اطلاعات با حجم بالا
\LTRfootnote{
Dynamic Aspect of Massive Data
} 
ناموفق عمل کرده است. یادگیری برخط، تلاشی است که فائق آمدن بر این مشکل. در ادامه‌ی این نوشته ما موضوع را بهینه‌سازی برخط نام خواهیم نهاد زیرا به پروتکل ارائه شده نزدیک‌تر است.



\begin{quote}
{\bf
پروتکل بهینه‌سازی برخط
}

یادگیری برخط گسترش طبیعی یادگیری آماری است. به تعبیری می‌توان یادگیری برخط را به عنوان روشی در مرزهای نتایج آزاد از توزیع، دید. در واقع تفاوت بنیادی میان یادگیری برخط و یادگیری آماری، به علاوه‌ی اینکه داده به شکل دنباله‌ای وجود دارد، این حقیقت است که فرض احتمالاتی‌ای در رابطه با دنباله‌ی داده 
$Z_1, \cdots, Z_n$ 
انجام نمی‌شود. با کاهش مسئله به ساختار مینیمال
\LTRfootnote{
Minimal
} 
آن، می‌توان امیدوار بود که به سختی بنیادی در مسئله‌های یادگیری رسید. همانگونه که مشخص شده است، این تغییر نگرش به مسئله، بسیار مثمر ثمر است و تاثیری عمیق و بنیادین بر چشم‌انداز نوین یادگیری ماشین گذاشته است.


پروتکل یادگیری برخط به شکل رسمی در ادامه آمده است، در هر گام از زمان 
$t = 1, 2, \cdots, n$: 

\begin{itemize}
\item
عمل 
$a_t \in \set{A}$ 
را انتخاب کنید

\item
همزمان یک فرد متخاصم
\LTRfootnote{
Adversary
} 
(یا طبیعت) انتخاب می‌کند
$z_t \in \randomvariable{Z}$

\item
ضرر شرایط کنونی را متحمل می‌شوید 
$\mthfnc{l}(a_t, z_t)$

\item
عمل فرد متخاصم (یا طبیعت) را مشاهده می‌کنید 
$z_t$

\end{itemize}

هدف کمینه کردن (و کنترل) پشیمانی
\LTRfootnote{
Regret
} 
تجمعی است:
\[
R_n = \sum^{n}_{i = 1} \mthfnc{l}(a_t,z_t) - \inf_{a \in \set{A}} \sum^{n}_{t=1} \mthfnc{l}(a,z_t)
\]

\end{quote}

در کلام، پشیمانی تجمعی مقدار ضرر تجمعی بازیکن را با ضرر تجمعی بهترین عمل (با اطلاعات موجود) مقایسه می‌کند. در اینجا تلاش برای یافتن کران‌هایی ( به طور مثلا کران‌های بالا برای 
$R_n$) 
مستقل از انتخاب‌های فرد متخاصم هستیم.

% Example
{\bf
مثال: رگرسیون برخط، دسته‌بندی
\LTRfootnote{
Classification
} 
برخط
} 

در هر گام، بازیکن یک تابع رگرسیون انتخاب می‌کند 
$\mthfnc{a}_t : \set{X} \rightarrow \set{Y}$ 
و متخاصم نیز یک جفت ورودی/خروجی 
$(x_t, y_t) \in \set{X} \times \set{Y}$ 
انتخاب می‌کند. بازیکن ضرر 
$\mthfnc{l}( \mthfnc{a}_t, (x_t, y_t) )$ 
را متحمل می‌شود و جفت 
$(x_t, y_t)$ 
را مشاهده می‌کند. به طور معمول مجموعه‌ی 
$\set{A}$ 
به خانواده‌ی کوچکی از تابع‌های رگرسیون محدود می‌شود، مانند ابر صفحه‌های تصمیم
\LTRfootnote{
Decision Hyperplane
} 
در الگویابی
\LTRfootnote{
Pattern Recognition
} 
برخط.



% Example
{\bf
مثال: سرمایه‌گذاری پی‌درپی
\LTRfootnote{
Sequential Investment
}
} 

یک بازار سهام ایده‌آل
\LTRfootnote{
Idealized Stock Market
} 
با 
$d$ 
دارایی
\LTRfootnote{
Asset
} 
را در نظر بگیرید. این بازار را با بردار 
$z \in \hollow{R}^d_+$ 
نمایش دهید که نشان دهنده‌ی قیمت‌های نسبی در یک بازه‌ی زمانی است. به این معنی که اگر سرمایه‌گذاری 
$a \in \hollow{R}^d_+$ 
انجام شود ( 
$a(i)$ 
مقدار در دارایی 
$i$ 
سرمایه‌گذاری شده)، برگشت سرمایه به اندازه‌ی 
$ \sum^{d}_{i=1} a(i) z(i) = a^T z$ 
خواهد بود.

حال به بررسی مسئله‌ی سرمایه‌گذاری پی‌درپی در این بازار سهام می‌پردازیم. در هر بازه‌ی معامله 
$t$، 
سرمایه
\LTRfootnote{
Capital
} 
کنونی را با 
$W_{t-1}$ 
نمایش دهید و بازیکن تمام سرمایه‌ی کنونی را با توجه به نسبت‌های 
$a_t \in \set{A} = \left\lbrace a \in \hollow{R}^d_+ , \sum^{d}_{i=1} a(i) = 1 \right\rbrace$ 
سرمایه گذاری می‌کند. همزمان بازار بردار 
$z_t \in \hollow{R}^d_+$ 
را مشخص می‌کند. ثروت در پایان دوره‌ی 
$t$ 
به شکل زیر است:
\[
W_t = \sum^{d}_{i=1} a_t (i) W_{t-1} z_t (i) = W_{t-1} a^T_t z_t = W_0 \prod^{t}_{s=1} a^T_s z_s
\]

یک خانواده‌ی مهم از استراتژی‌های سرمایه‌گذاری در این مسئله، مجموعه‌ی پورتفولیوهای متعادل شونده ثابت
\LTRfootnote{
Constantly Rebalanced Portfolio
} 
هستن، که به مجموعه‌ی تصمیم‌هایی به شکل 
$a_t = a, \forall t \geq 1$ 
تعلق دارند. به بیان دیگر، بازی‌کن در هر بازه‌ی معامله‌ی 
$t$ 
سرمایه‌ی خود 
$W_{t-1}$ 
را با توجه به نسبت‌های 
$a$ 
تقسیم می‌کند.

هدف مسئله، قابل رقابت بودن با این خانواده از استراتژی‌ها است که می‌توان آن را با ثروت نسبی رقابتی
\LTRfootnote{
Competetive Wealth Ratio
} 
سنجید:
\[
\sup_{a \in \set{A}} \left\lbrace \frac{ W^a_n }{ W_n } = \frac{ W_0 \prod^{n}_{s=1} a^T z_s }{ W_n } \right\rbrace
\]

بنابراین برای ثروت نسبی لگاریتمی
\LTRfootnote{
Logarithmic Wealth Ratio
} 
خواهیم داشت:
\[
\sum^{n}_{t=1} - \log ( a^T_t z_t ) - \inf_{a \in \set{A}} \sum^{n}_{t=1} - \log (a^T z_t)
\]
که در واقع پشیمانی تجمعی برای مسئله‌ی بهینه‌سازی برخط بر روی سادک
\LTRfootnote{
Simplex
} 
$d-1$ 
بعدی با تابع هزینه‌ی لگاریتمی 
$\mthfnc{l} (a,z) = -\log (a^T z)$ 
است.
















































% --> Concept Drift
\subsubsection{
رانش مفهوم
}































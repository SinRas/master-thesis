% !TEX encoding = UTF-8 Unicode
\documentclass[a4paper,11px]{article}

% ------ Base Path ----- %
\newcommand{\basepath}{../}

% ------- Style Packages ------------------ %
% !TEX encoding = UTF-8 Unicode
% -- Clear Page Before Each Section --- %
\usepackage{titlesec}
\newcommand{\sectionbreak}{\clearpage}

% ------- Hyperref Links ------------- %
\usepackage{import}

% ------- Hyperref Links ------------- %
\usepackage[colorlinks=true]{hyperref}

% ------- Mathematical Symbols ------- %
\usepackage{amsmath, amssymb, mathrsfs}

% ------- Mathematical Theorems ------- %
\usepackage{amsthm}

% ------- Pseudo Code ------- %
\usepackage{algorithmic, algorithm, caption}

% ------- Reset Footnote Numbers each Page ------- %
\usepackage[perpage]{footmisc}

% ------- XePersian ------------------ %
\usepackage[extrafootnotefeatures]{xepersian}

% ------ Font Styles ---------------- %
\settextfont[Scale=1]{XBNiloofar}
\setdigitfont{PGaramond}

% ------- Commands ------------------ %
\input{../commands.cmd}



\begin{document}

% ---------------------------------------
% Appendices
\section{
ضمیمه‌ها
}

ضمیمه‌ها در اینجا قرار می‌گیرند.

% ---------------------------------------
% Potential Taylor: Appendix Proof
\subsection{
اثبات قضیه‌ی 
\ref{potentialtaylor}
}
ابتدا بسط تایلور تابع 
$\fPhi$ 
را در نقطه‌ی 
$\vect{R}_{t-1}$ 
می‌نویسیم و آن را در نقطه‌ی 
$\vect{R}_{t}$ 
حساب می‌کنیم.
\[
\begin{split}
&\fPhi( \vect{R}_{t} ) = \fPhi( \vect{R}_{t-1} + \vect{r}_t )\\
&= \fPhi( \vect{R}_{t-1} ) + \nabla \fPhi( \vect{R}_{t-1} ) + \frac{1}{2} \sum^{N}_{i=1} \sum^{N}_{j=1} \frac{ \partial^2 \fPhi }{ \partial u_i \partial u_j } \vert_{\xi} r_{i,t} r_{j,t}\\
& ( \mbox{ where } \xi \mbox{ is some vector in } \hollow{R}^N )\\
&\leq \fPhi( \vect{R}_{t-1} ) + \frac{1}{2} \sum^{N}_{i=1} \sum^{N}_{j=1} \frac{ \partial^2 \fPhi }{ \partial u_i \partial u_j } \vert_{\xi} r_{i,t} r_{j,t}\\
& ( \mbox{ using Blackwell's Condition } )
\end{split}
\]
حال با استفاده از تعریف تابع 
$\fPhi$:
\[
\begin{split}
&\sum^{N}_{i=1} \sum^{N}_{j=1} \frac{\partial^2 \fPhi}{\partial u_i \partial u_j}\vert_{\xi} r_{i,t} r_{j,t}\\
&= \fpsi^{\prime\prime} \left( \Sum{N}{i=1} \fphi(\xi_i) \right) \Sum{N}{i=1} \Sum{N}{j=1} \fphi^\prime (\xi_i) \fphi^\prime (\xi_j) r_{i,t} r_{j,t}\\
& + \fpsi^\prime \left( \Sum{N}{i=1} \fphi(\xi_i) \right) \Sum{N}{i=1} \fphi^{\prime\prime} (\xi_i) r^{2}_{i,t}\\
&= \fpsi^{\prime\prime} \left( \Sum{N}{i=1} \fphi(\xi_i) \right) \left( \Sum{N}{i=1} \fphi^{\prime} (\xi_i) r_{i,t} \right)^2 + \fpsi^\prime \left( \Sum{N}{i=1} \fphi( \xi_i ) \right) \Sum{N}{i=1} \fphi^{\prime\prime} (\xi_i) r^{2}_{i,t}\\
& \leq \fpsi^\prime \left( \Sum{N}{i=1} \fphi( \xi_i ) \right) \Sum{N}{i=1} \fphi^{\prime\prime} (\xi_i) r^{2}_{i,t} \hspace{1cm} \mbox{since } \fpsi \mbox{ is concave}\\
& \leq \mthfnc{C} (\vect{r}_t)
\end{split}
\]
که در گام آخر از تعریف 
$\mthfnc{C} (\vect{r}_t)$ 
که در قضیه‌ی
\ref{potentialtaylor} 
آمده، استفاده شده است. پس بدست آوردیم که در هر گام خواهیم داشت:
\[
\fPhi (\vect{R}_{t}) - \fPhi (\vert{R}_{t-1}) \leq \frac{1}{2} \mthfnc{C} (\vect{r}_t)
\]
صورت قضیه با جمع بستن بر روی زمان 
$t = 1, \cdots, n$ 
بدست می‌آید.


























\end{document}
% !TEX encoding = UTF-8 Unicode


% ---------------------------------------
% Our Approach
\section{
روش پایان‌نامه
}

% ---------------------------------------
% The research method by which you investigate the world.
% - a short summary of the available methods
% - your choice
% - detailed report of how you actually carried out your research. Presenting how you seleced the people taking part is of special importance.


در این پایان‌نامه روشی که انتخاب شده، استفاده از 
\textit{
طالع‌بین با وزن‌دهی نمایی
} 
به همراه سازوکارهایی برای تشخیص و تطبیق با رانش مفهوم، است. ابتدا مسئله به شکل شفاف به همراه چند مثال بیان می‌شود تا نیازها بررسی شود.

% ---------------------------------------
% Practical Challenges
\subsection{
چالش‌های عملی
}

مشاهده‌ها و چالش‌هایی در عمل در استفاده از طالع‌بین با وزن‌دهی نمایی وجود دارد که انگیزه برای بررسی و استفاده از رانش مفهوم است. در ادامه به چند چالش پرداخته می‌شود که در مثال‌های عملی ظاهر می‌شوند.

% ---------------------------------------
% Practical Challenges:‌ Stock
\subsubsection{
پیش‌بینی قیمت سهام
}



همان گونه که در شکل 
\ref{fig:wmstatic} 
مشاهده می‌کنید، با فرض 
\textit{
ایستا
}\LTRfootnote{
Static
} 
بودن عملکرد متخصص‌ها (به این معنی که یک متخصص از دیگران همواره بهتر عمل می‌کند)، پیش‌بینی طالع‌بین با وزن‌دهی نمایی به سرعت به متخصص آبی رنگ همگرا می‌شود. این امر باعث می‌شود که از زمان حدودا 50 به بعد، این متخصص به نظر اصلی تبدیل شود و پیش‌بینی طالع‌بین برابر پیش‌بینی او باشد. همان‌گونه که در نمودار مشاهده می‌کنید، در زمان 400 یک تغییر نظام رخ می‌دهد که در آن، عملکرد دو متخصص سبز و نارنجی، بهتر شده است اما به دلیل حافظه و گذشته‌ای که در وزن‌های متخصص‌ها نهفته است، نظر متخصص آبی اکثریت خواهد ماند و این باعث می‌شود که مدل توانایی استفاده از نظر متخصص‌هایی که به طور موضعی خوب عمل می‌کنند را نداشته باشد.


\begin{figure}[h!]
\includegraphics[width=\linewidth]{\images test_expert_weights.png}
\caption{
نمودار وزن‌های چند متخصص
}
\label{fig:wmstatic}
\end{figure}




% ---------------------------------------
% Required Features
\subsection{
ویژگی‌های مورد نیاز
} 
به منظور مواجهه با چالش‌های مطرح شده در قسمت قبل، ویژگی‌های زیر برای یک روش جدید مورد تصور است:
\begin{itemize}
\item \textbf{
تطبیق سریع
}: 
در صورتی که یک یا چند متخصص عملکرد خوبی از خود ارائه کنند، به سرعت شناسایی شوند تا از عملکرد خوب آن‌ها (هرچند در بازه‌ای کوتاه) استفاده شود.
\item \textbf{
انعطاف پذیری بالا
}: 
اگر یک یا چند متخصص که در گذشته عملکرد خوبی از خود نشان نداده‌اند (و حتی عملکرد بسیار بدی از خود به نمایش گذاشتن)، در حال نشان دادن عملکرد خوب بودند (به معنی اینکه رانش مفهوم رخ داده)، بتوان از عملکرد آن‌ها استفاده کرد.

\item \textbf{
حساس نبودن به نویز
}: 
نویز را می‌توان به شکل تغییر موضعی و با دامنه‌ی کوچک در رفتار مسئله تعبیر کرد. مدل پیش‌نهادی نیازمند این است که در برابر نویز، پایدار باشد و با تغییرات کوچک در خطا و ورودی، تغییرات بسیار بزرگ در متغیرهای مسئله نتیجه ندهد.

\end{itemize}

با توجه به این‌که روش اصلی مورد بررسی، استفاده از «طالع‌بین با وزن‌دهی نمایی» است، بنای تحلیل‌ها را بر این نوع مدل می‌گذاریم و ویژگی‌های زیر را برای ارائه‌ی روش خود فرض می‌کنیم:
\begin{enumerate}
\item \textbf{
تنها وزن متخصص‌ها
}: 
با توجه به این‌که تحلیل رفتار دنباله‌ی رویدادها 
$y_1, y_2, \cdots \in \set{Y}$ 
نیازمند دانش نسبت به مسئله‌ی مورد بررسی است. ما سعی می‌کنیم که رفتار وزن متخصص‌ها را بررسی کنیم که به معنایی مستقل از مسئله‌ی مورد بررسی است و تنها می‌خواهیم که نتیجه‌گیری با استفاده از رفتار متخصص‌ها را بهینه کنیم. یعنی به بررسی نقاط درون 
\textit{
سادک
}\LTRfootnote{
Simplex
} 
وزن‌های متخصص‌ها می‌پردازیم، که دنباله‌ی 
$x_1, x_2, \cdots \in \simplex{N}{set} $ 
است.

\item \textbf{
تنها متخصص‌های داده‌شده
}: 
فرض می‌کنیم که مجموعه‌ی متخصص‌های داده شده، تغییر نمی‌کند و تنها با استفاده از یک مجموعه‌ی از ابتدا دانسته شده از متخصص‌ها کار خواهیم کرد. می‌توان تعدادی از آن‌ها را در تصمیم طالع‌بین، وارد نکرد (به طور مثال وزن آن‌ها را در تصمیم‌گیری صفر قرار داد) اما مجموعه‌ی متخصص‌ها کوچک نمی‌شود. همچنین متخصص جدیدی ایجاد نمی‌شود (با فرض بی اطلاع بودن از مفهوم مورد بررسی).

\item \textbf{
*** چیز دیگری؟***
}

\end{enumerate}

نگاه اصلی در پس ویژگی‌های بالا، این است که تمامی اطلاعات در دست روش، اطلاعات مربوط به عملکرد متخصص‌ها خواهد بود. به این معنی که از اطلاعات به دست آمده از فضای رویدادها، استفاده نمی‌شود. برای واضح شدن نگرش به یک مثال توجه کنید.
\begin{quote}
فرض کنید که دو متخصص در اختیار دارید 
$\set{E} = \{ e_1, e_2 \}$. 
متخصص اول تمامی پیش‌بینی‌های خود را به درستی انجام می‌دهد و متخصص دوم برخی پیش‌بینی‌های خود را درست و برخی را غلط انجام می‌دهد. در این صورت، هرچند که رویدادهای مشاهده شده، دچار تغییر و رانش مفهوم شوند، اگر متخصص اول به خوبی توان پیش‌بینی آن‌ها را داشته باشد، در نوع استدلال شما (وزن‌دهی به متخصص‌ها)، تغییری رخ نخواهد داد و از منظر شما، رانش مفهوم دیده نخواهد شد.
\end{quote}

تغییر مفهوم پنهان در این نوع نگرش، تغییر معنادار نقطه‌ی 
$x_t \in \simplex{N}{}$ 
خواهد بود. با فرض ثابت بودن تعداد متخصص‌ها، تنها 
\textit{
پویایی
}\LTRfootnote{
Dynamic
} 
سیستم مورد بررسی، ناشی از حرکت وزن متخصص‌ها در فضا بوده و پویایی ناشی از تغییر فضا وجود نخواهد داشت.

% ---------------------------------------
% Solution
\subsection{
راه حل
}
به منظور مقابله با چالش‌های عملی مطرح شده، ساز و کار زیر پیش‌نهاد می‌شود:
\begin{quote}
از وزن‌دهی نمایی برای تصمیم‌گیری بر اساس متخصص‌ها استفاده شود به همراه سازوکاری که در صورت بروز رانش مفهوم، وزن‌دهی را تسریع کند.
\end{quote}

بگذارید که در ابتدا روش را به شکل کلامی توضیح دهیم:
\begin{itemize}
\item 
مشاهده‌ای از محیط به متخصص‌ها می‌رسد
\item 
هر متخصص پیش‌بینی خود را گزارش می‌کند
\item 
با توجه به وزن‌های موجود و پیش‌بینی‌های متخصص‌ها، طالع‌بین تصمیم می‌گیرد
\item 
محیط میزان ضرر تصمیم گرفته شده توسط طالع‌بین (و هر متخصص) را گزارش می‌کند
\item 
وزن‌های هر متخصص بر اساس روش به‌روزکردن نمایی و همچنین تشخیص وجود تغییر معنادار در خطای هر متخصص، به روز می‌شود
\end{itemize}

حال به نوشتن شبه‌کد برای الگوریتم می‌پردازیم و سپس هر قسمت را توضیح می‌دهیم:
\begin{algorithm}[H]
\caption{
وزن‌دهی نمایی به همراه رانش مفهوم
}
\begin{algorithmic}[1]
\REQUIRE
محیط، لیست متخصص‌ها، پارامترهای زیرالگوریتم‌ها\\
وزن‌دهنده، استنتاج کننده، محاسبه‌کننده‌ی ضرر
  \STATE
  مقداردهی 
  \textit{
  وزن‌دهنده
  }
  (وزن‌های متخصص‌ها)
  \STATE 
  مشاهده‌ی گذشته <- پوچ
  \WHILE {$1 == 1$}
  \STATE 
  \# \textbf{
  تصمیم‌گیری
  }
  \STATE 
  بردار پیش‌بینی‌ها <- اطلاع مشاهده‌ی گذشته به متخصص‌ها و گرفتن بردار پیش‌بینی‌ها
  \STATE 
  بردار وزن‌های کنونی <- گرفتن بردار وزن‌ها از 
  \textit{
  وزن‌دهنده
  }
  \STATE
  تصمیم طالع‌بین <- رای‌گیری اکثریت بر اساس بردار وزن‌های کنونی و بردار پیش‌بینی‌ها
  \STATE 
  \# \textbf{
  تحول محیط
  }
  \STATE 
  مشاهده‌ی جدید <- تحول 
  \textit{
  محیط
  } 
  با توجه به تصمیم طالع‌بین و گزارش وضعیت جدید
  \STATE 
  \# \textbf{
  به‌روز رسانی وزن‌ها
  }
  \STATE 
  بردار ضرر <- محاسبه‌ی بردار خطاها بر اساس مشاهده‌ی گذشته، بردار پیش‌بینی‌ها و مشاهده‌ی جدید
  \STATE 
  بردار تغییرها <- استنتاج براساس تاریخچه‌ی بردارهای ضرر (به علاوه‌ی بردار ضرر کنونی)
  \STATE 
  به‌روز رسانی 
  \textit{
  وزن‌دهنده
  } 
  بر اساس بردار ضرر و بردار تغییرها
  \STATE 
  \# \textbf{
  پیش‌بردن زمان به جلو
  }
  \STATE 
  مشاهد‌ی گذشته <- مشاهده‌ی جدید
  \ENDWHILE
\end{algorithmic}
\end{algorithm}

در الگوریتم ارائه شده در بالا، چند زیرروش وجود دارد:
\begin{itemize}
\item \textbf{
محاسبه‌کننده‌ی ضرر
}: نقش محاسبه‌ی ضرر
\LTRfootnote{
Loss
} 
با استفاده از مشاهده‌ی گذشته، مشاهده‌ی کنونی و بردار پیش‌بینی‌ها را بر عهده دارد. که به طور مثال می‌تواند میزان مطلق خطای پیش‌بینی نسبت به واقعیت باشد.

\item \textbf{
استنتاج کننده
}: استنتاج کننده وظیفه دارد تا با استفاده از تاریخچه‌ی خطاهای متخصص‌ها، اعلام کند که در لحظه‌ی کنونی کدام متغیرها دستخوش تغییر قابل ملاحضه (با در نظر گرفتن علامت) شده اند.

\item \textbf{
وزن‌دهنده
}: وزن‌دهنده از ترکیب وزن‌دهی نمایی (جریمه کردن بر اساس میزان خطا) و همچنین شتاب دادن به تغییر وزن‌ها (بر اساس میزان بردار تغییرها)، وزن‌های جدید را محاسبه می‌کند.
\end{itemize}

در روش ارائه شده از زیرروش‌های زیر استفاده شده است.


% ---------------------------------------
% Solution: Loss
\subsubsubsection{
محاسبه‌ی ضرر
}

در مسئله‌ی مورد بررسی، تلاش بر پیش‌بینی رفتار قیمت یک سهم است. رفتار قیمت هر سهم در هر روز به شکل یک متغیر تصادفی 
$\set{Y} = \{ -1, 0, 1 \}$ 
توصیف می‌شود که در آن مقدار 
$1$ 
نمایانگر افزایش قیمت نسبت به روز گذشته، 
$0$ 
تغییر نکردن قیمت نسبت به روز گذشته، و 
$-1$ 
کاهش قیمت نسبت به روز گذشته است.

بردار ضررها را با 
$\vect{l}$ 
نمایش می‌دهیم. به منظور در نظر گرفتن ترتیب میان پیش‌بینی‌ها، از قدر مطلق اختلاف نسبت به مقدار واقعی استفاده می‌شود:
\[
\vect{l}(i) = | f_{E,i} - y_i |
\]
در این صورت نزدیکی مقدار پیش‌بینی و مقدار واقعی نیز در خطا در نظر گرفته می‌شود. به این معنی که پیش‌بنی مقدار 
$-1$ 
و رخ‌دادن مقدار 
$1$ 
یک خطای دو واحدی را نتیجه می‌دهد.


% ---------------------------------------
% Solution: Inference
\subsubsubsection{
استنتاج کننده
}

به منظور شناسایی تغییر، نیاز به استنتاج با توجه به داده‌ها است. طبق فرض‌های بیان شده، تنها به عملکرد متخصص‌ها نگاه می‌شود به طور صریح به توزیع رویدادها پرداخته نمی‌شود. همان‌گونه که در بخش ***شناسایی تغییر*** بیان شد، به منظور تشخیص تغییر، روش‌های گوناگونی وجود دارد.

در روش ارائه شده به منظور تشخیص تغییر در عملکرد هر متخصص، از آزمون کولموگروف-اسمیرنوف
\LTRfootnote{
Kolmogorov-Smirnov Test
} 
بر روی دو پنجره‌ی داده از خطاهای متخصص استفاده می‌شود. یک پنجره‌ی کوچک و متاخر، به همراه یک پنجره‌ی بزرگ و قدیمی‌تر که اشتراکی با یکدیگر ندارند. پی-مقدار
\LTRfootnote{
P-Value
} 
مقایسه‌ی این دو پنجره گزارش می‌شود.

% ---------------------------------------

سپس از تصحیح بنجامینی-هوشبرگ
\LTRfootnote{
Benjamini-Hochberg Correction
} 
(با فرض مستقل بودن عملکرد متخصص‌ها) برای تصحیح بردار پی-مقدارهای متخصص‌های مختلف استفاده می‌شود و با توجه به آستانه‌ی در نظر گرفته شده، تعدادی از پی-مقدارهای تصحیص شده را به عنوان قابل توجه انتخاب می‌کنیم. علامت یک تغییر قابل توجه با توجه به تغییر میانگین دو پنجره انتخاب می‌شود.
\textit{
بردار تغییر
} 
$\vect{c}$ 
دارای مولفه‌های 
$\{ -1, 0, 1 \}$ 
است که مقدار 
$0$ 
بیان کننده‌ی نبود تغییر چشم‌گیر، و مقدار 
$+1$ 
نمایانگر تغییر چشم‌گیر در راستای بهبود عملکرد، و مقدار 
$-1$ 
نمایانگر تغییر چشم‌گیر در راستای بدتر شدن عملکرد است.


% ---------------------------------------

\subsubsubsection{
وزن‌دهنده
}

وزن‌دهنده یک قسمت پایه دارد، که از روش به‌روز رسانی نمایی با پارامتر داده‌شده 
$\beta$، 
استفاده می‌کند. سپس یک تصحیح با توجه به مقدار بردار تغییرها 
$\vect{c}$ 
و یک پارامتر مقیاس 
$\gamma$ 
انجام می‌شود. در واقع از خطای جدید به شکل استفاده می‌شود:
\[
\vect{l}^{\prime}(i) = \vect{l}(i) - \gamma \vect{c}(i)
\]
و سپس به روز کردن وزن‌ها با استفاده از بردار ضررهای جدید 
$\vect{l}^\prime$ 
انجام می‌شود:
\[
w_{i,t+1} = w_{i,t} * \exp( -\beta \vect{l}^{\prime}(i) )
\]

% ---------------------------------------

پس از بیان روش، حال به بیان ابرپارامترها
\LTRfootnote{
Hyper-Parameters
} 
و تحلیل ارتباط‌های میان آن‌ها می‌پردازیم.
\begin{itemize}
\item \textbf{
اندازه‌ی پنجره‌ی متاخر
}

\item \textbf{
اندازه‌ی پنجره‌ی گذشته
}

\item \textbf{
آستانه در روش بنجامینی-هوشبرگ
}

\item \textbf{
نرخ اثر خطا
}

\item \textbf{
نرخ اثر تغییر
}
\end{itemize}













































% !TEX encoding = UTF-8 Unicode
\documentclass[a4paper,11px]{article}

% ------ Base Path ----- %
\newcommand{\basepath}{../}

% ------- Style Packages ------------------ %
% !TEX encoding = UTF-8 Unicode
% -- Clear Page Before Each Section --- %
\usepackage{titlesec}
\newcommand{\sectionbreak}{\clearpage}

% ------- Hyperref Links ------------- %
\usepackage{import}

% ------- Hyperref Links ------------- %
\usepackage[colorlinks=true]{hyperref}

% ------- Mathematical Symbols ------- %
\usepackage{amsmath, amssymb, mathrsfs}

% ------- Mathematical Theorems ------- %
\usepackage{amsthm}

% ------- Pseudo Code ------- %
\usepackage{algorithmic, algorithm, caption}

% ------- Reset Footnote Numbers each Page ------- %
\usepackage[perpage]{footmisc}

% ------- XePersian ------------------ %
\usepackage[extrafootnotefeatures]{xepersian}

% ------ Font Styles ---------------- %
\settextfont[Scale=1]{XBNiloofar}
\setdigitfont{PGaramond}

% ------- Commands ------------------ %
\input{../commands.cmd}



\begin{document}

% ---------------------------------------
% Our Approach
\section{
روش پایان‌نامه
}

% ---------------------------------------
% The research method by which you investigate the world.
% - a short summary of the available methods
% - your choice
% - detailed report of how you actually carried out your research. Presenting how you seleced the people taking part is of special importance.


در این پایان‌نامه روشی که انتخاب شده، استفاده از 
\textit{
طالع‌بین با وزن‌دهی نمایی
} 
به همراه سازوکارهایی برای تشخیص و تطبیق با رانش مفهوم، است. ابتدا مسئله به شکل شفاف به همراه چند مثال بیان می‌شود تا نیازها بررسی شود.

% ---------------------------------------
% Practical Challenges
\subsection{
چالش‌های عملی
}

مشاهده‌ها و چالش‌هایی در عمل در استفاده از طالع‌بین با وزن‌دهی نمایی وجود دارد که انگیزه برای بررسی و استفاده از رانش مفهوم است. در ادامه به چند چالش پرداخته می‌شود که در مثال‌های عملی ظاهر می‌شوند.

% ---------------------------------------
% Practical Challenges:‌ Stock
\subsubsection{
پیش‌بینی قیمت سهام
}



همان گونه که در شکل 
\ref{fig:wmstatic} 
مشاهده می‌کنید، با فرض 
\textit{
ایستا
}\LTRfootnote{
Static
} 
بودن عملکرد متخصص‌ها (به این معنی که یک متخصص از دیگران همواره بهتر عمل می‌کند)، پیش‌بینی طالع‌بین با وزن‌دهی نمایی به سرعت به متخصص آبی رنگ همگرا می‌شود. این امر باعث می‌شود که از زمان حدودا 50 به بعد، این متخصص به نظر اصلی تبدیل شود و پیش‌بینی طالع‌بین برابر پیش‌بینی او باشد. همان‌گونه که در نمودار مشاهده می‌کنید، در زمان 400 یک تغییر نظام رخ می‌دهد که در آن، عملکرد دو متخصص سبز و نارنجی، بهتر شده است اما به دلیل حافظه و گذشته‌ای که در وزن‌های متخصص‌ها نهفته است، نظر متخصص آبی اکثریت خواهد ماند و این باعث می‌شود که مدل توانایی استفاده از نظر متخصص‌هایی که به طور موضعی خوب عمل می‌کنند را نداشته باشد.


\begin{figure}[h!]
\includegraphics[width=\linewidth]{\images test_expert_weights.png}
\caption{
نمودار وزن‌های چند متخصص در زمان در مسئله‌ی پیش‌بینی قیمت یک سهم. همانگونه که مشاهده می‌کنید.
}
\label{fig:wmstatic}
\end{figure}




% ---------------------------------------
% Our Approach
\subsection{
راه‌حل
} 
به منظور مواجهه با چالش‌های مطرح شده در قسمت قبل، ویژگی‌های زیر برای یک روش جدید مورد تصور است:
\begin{itemize}
\item \textbf{
تطبیق سریع
}: 
در صورتی که یک یا چند متخصص عملکرد خوبی از خود ارائه کنند، به سرعت شناسایی شوند تا از عملکرد خوب آن‌ها (هرچند در بازه‌ای کوتاه) استفاده شود.
\item \textbf{
انعطاف پذیری بالا
}: 
اگر یک یا چند متخصص که در گذشته عملکرد خوبی از خود نشان نداده‌اند (و حتی عملکرد بسیار بدی از خود به نمایش گذاشتن)، در حال نشان دادن عملکرد خوب بودند (به معنی اینکه رانش مفهوم رخ داده)، بتوان از عملکرد آن‌ها استفاده کرد.

\item \textbf{
حساس نبودن به نویز
}: 
نویز را می‌توان به شکل تغییر موضعی و با دامنه‌ی کوچک در رفتار مسئله تعبیر کرد. مدل پیش‌نهادی نیازمند این است که در برابر نویز، پایدار باشد و با تغییرات کوچک در خطا و ورودی، تغییرات بسیار بزرگ در متغیرهای مسئله نتیجه ندهد.

\end{itemize}

با توجه به این‌که روش اصلی مورد بررسی، استفاده از «طالع‌بین با وزن‌دهی نمایی» است، بنای تحلیل‌ها را بر این نوع مدل می‌گذاریم و ویژگی‌های زیر را برای ارائه‌ی روش خود فرض می‌کنیم:
\begin{enumerate}
\item \textbf{
تنها وزن متخصص‌ها
}: 
با توجه به این‌که تحلیل رفتار دنباله‌ی رویدادها 
$y_1, y_2, \cdots \in \set{Y}$ 
نیازمند دانش نسبت به مسئله‌ی مورد بررسی است. ما سعی می‌کنیم که رفتار وزن متخصص‌ها را بررسی کنیم که به معنایی مستقل از مسئله‌ی مورد بررسی است و تنها می‌خواهیم که نتیجه‌گیری با استفاده از رفتار متخصص‌ها را بهینه کنیم. یعنی به بررسی نقاط درون 
\textit{
سادک
}\LTRfootnote{
Simplex
} 
وزن‌های متخصص‌ها می‌پردازیم، که دنباله‌ی 
$x_1, x_2, \cdots \in \simplex{N}{set} $ 
است.

\item \textbf{
تنها متخصص‌های داده‌شده
}: 
فرض می‌کنیم که مجموعه‌ی متخصص‌های داده شده، تغییر نمی‌کند و تنها با استفاده از یک مجموعه‌ی از ابتدا دانسته شده از متخصص‌ها کار خواهیم کرد. می‌توان تعدادی از آن‌ها را در تصمیم طالع‌بین، وارد نکرد (به طور مثال وزن آن‌ها را در تصمیم‌گیری صفر قرار داد) اما مجموعه‌ی متخصص‌ها کوچک نمی‌شود. همچنین متخصص جدیدی ایجاد نمی‌شود (با فرض بی اطلاع بودن از مفهوم مورد بررسی).

\item \textbf{
*** چیز دیگری؟***
}

\end{enumerate}

نگاه اصلی پشت ویژگی‌های بالا، این است که تمامی اطلاعات در دست روش، اطلاعات مربوط به عملکرد متخصص‌ها خواهد بود. به این معنی که از اطلاعات به دست آمده از فضای رویدادها، استفاده نمی‌شود. برای واضح شدن نگرش به یک مثال توجه کنید.
\begin{quote}
فرض کنید که دو متخصص در اختیار دارید 
$\set{E} = \{ e_1, e_2 \}$. 
متخصص اول تمامی پیش‌بینی‌های خود را به درستی انجام می‌دهد و متخصص دوم برخی پیش‌بینی‌های خود را درست و برخی را غلط انجام می‌دهد. در این صورت، هرچند که رویدادهای مشاهده شده، دچار تغییر و رانش مفهوم شوند، اگر متخصص اول به خوبی توان پیش‌بینی آن‌ها را داشته باشد، در نوع استدلال شما (وزن‌دهی به متخصص‌ها)، تغییری رخ نخواهد داد و از منظر شما، رانش مفهوم دیده نخواهد شد.
\end{quote}

تغییر مفهوم پنهان در این نوع نگرش، تغییر معنادار نقطه‌ی 
$x_t \in \simplex{N}{}$ 
خواهد بود. با فرض ثابت بودن تعداد متخصص‌ها، تنها 
\textit{
پویایی
}\LTRfootnote{
Dynamic
} 
سیستم مورد بررسی، ناشی از حرکت وزن متخصص‌ها در فضا بوده و پویایی ناشی از تغییر فضا وجود نخواهد داشت.







































\end{document}